\subsection{Identification et affectation des tâches}
\vspace{2cm}
L'analyse effectuée grâce aux diagrammes WBS et GANTT met en avant deux directions principales:
\\
\begin{itemize}
\item création d'un cadre de mesure comportant l'électronique embarqué
\item traitement des différents flux d'informations pour afficher les données des ruches sur un site web\\
\end{itemize}

Nous n'avons pas souhaité scinder totalement le groupe selon ces deux directions pour ne pas perdre totalement contact avec une partie du projet. Certains se sont, cependant, spécialisés dans un domaine. Etiene s'est intéressée à la base de donnée et au site web. Nicolas s'est spécialisé dans les fonctionnalités de la carte Arduino. Benoit s'est spécialisé dans la confection du cadre de mesure. Les autres membres ont gardés des fonctions transverses, pouvant ainsi s'adapter à la situation et travailler ce qui était nécessaire.
  
\subsection{Méthodes de travail}
\vspace{2cm}
A chaque début de séance, nous nous retrouvons pour une petite réunion (10 à 15 minutes). Celle-ci a pour but d'informer tous les membres du groupe de l'avancée du travail et des problèmes rencontrés par chacun. Nous déterminons également les éléments qui doivent être effectués lors de cette séance et les membres qui en sont chargés. Enfin c'est une occasion de maintenir la cohésion dans le groupe.
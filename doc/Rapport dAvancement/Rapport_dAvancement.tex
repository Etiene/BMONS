% chercher des documents LaTeX dans styles, corps et bib
\makeatletter\def\input@path{{styles/}{corps/}{bib/}}\makeatother

% Utiliser le style rapport.cls
\documentclass{rapport}

\title{BMONS - Beehive Monitoring System} % titre du document
%\subtitle{Rapport davancement}
\author{Danckaers, Alice \\[3ex] Dalcol, Etiene \\[3ex] Nguyen, Nicolas Van-Nh�n \\[3ex] Raymond, Beno�t \\[3ex] Sellier, Armand \\[3ex] Zheng, Tao}
\date{\today}
\doctype{Rapport d'Avancement} 
\promo{Version 0.1}
\etablissement{\textsc{Ensta} Bretagne\\2, rue Fran�ois Verny\\
  29806 \textsc{Brest} cedex\\\textsc{France}\\Tel +33 (0)2 98 34 88 00\\ \url{www.ensta-bretagne.fr}}
\logoEcole{\includegraphics[height=4.2cm]{logo_ENSTA_Bretagne_Vertical_CMJN}}

% creer le glossaire
\makeglossaries
% creer l'index
\makeindex

\begin{document}
\inputencoding{latin1}
% inclure la liste des acronymes
\newglossaryentry{enstab}{type=glo,
  text={\textsc{ensta} Bretagne},
  name={\textsc{Ensta} Bretagne},
  description={�cole Nationale Sup�rieure de Techniques Avanc�es Bretagne}}
\newglossaryentry{uv}{type=glo,
  text={\textsc{uv}},
  name={\textsc{Uv}},
  description={Unit� de Valeur}}
\newglossaryentry{wysiwyg}{type=glo,
  text={\textsc{wysiwyg}},
  name={\textsc{Wysiwyg}},
  description={What You See Is What You Get}}

%%% Local Variables: 
%%% mode: latex
%%% TeX-master: "../guide"
%%% End: 

% creer le titre ici
\maketitle

% chargement du fichier abstract.tex

\section*{R�sum�}
  Dans le cadre de ses fonctions, l'ing�nieur se doit de ma�triser les
  techniques de base de r�daction de documents.

  Le pr�sent guide fournit les premiers �l�ments visant � la r�daction d'un
  rapport de projet. Il est destin� aux �l�ves effectuant un projet dans le
  cadre de leur scolarit� � l'\gls{enstab}.  Sont pr�sent�s, de fa�on
  synth�tique, les bases sur la nature et la structure du rapport, ses r�gles
  de composition et de typographie.

\section*{Mots cl�s}
Rapport de projet, m�moire, r�daction.

% passage aux conventions typologiques anglaises
\selectlanguage{english}
\section*{Abstract}
  In the context of their profession, engineers have to master basic
  techniques for writing documents.
  
  This guide provides the fundamental elements for writing a project
  report. It is intended for students undertaking projects during their
  studies at \gls{enstab} and it gives an overview of the basics in
  relation to the nature and components of a report, with formatting
  guidelines and typographic rules.
  
\section*{Keywords}
Project report, memoir, writing.

% retour aux conventions fran�aises
\selectlanguage{french}


%%% Local Variables: 
%%% mode: latex
%%% TeX-master: "../guide"
%%% End: 


\clearpage

% table des matieres
\tableofcontents

\clearpage
% chargement du fichier corps.tex
\section{Introduction}

La structure de ce document et certaines parties sont inspir�es de
\autocite{guide}.

L'objectif de ce guide est de donner un cadre � l'�criture de rapports 
bibliographiques dans le cadre de l'\gls{uv}1.4
� l'\gls{enstab}.

La section \ref{sec:gen} pr�sente les g�n�ralit�s sur la nature d'un rapport
de projet. La section \ref{sec:struct} d�crit la structure attendue d'un rapport de
projet.
Enfin, la section
\ref{sec:outils} offre un aper�u des outils contribuant � l'�criture d'un rapport.


\section{G�n�ralit�s}
\label{sec:gen}

En r�digeant un rapport, vous laissez une trace de votre travail, qui restera
disponible sur le long terme. Il faudra donc veiller � la qualit� du fond
comme de la forme.

Afin d'�tre largement compris, le rapport doit privil�gier une
�criture simple (mais non simpliste) faite de phrases courtes,
employant un vocabulaire explicite. Tous les acronymes doivent �tre explicit�s
et tous les termes techniques expliqu�s.

L'orthographe contribue � l'image que vous laissez de vous-m�me. Si les
correcteurs orthographiques sont d'un usage indispensable, ils ne remplacent
jamais une relecture soigneuse.
De plus, les usages typographiques de la langue employ�e doivent �tre respect�s.


\section{Structure du rapport}
\label{sec:struct}

Il n'y a pas de forme unique et universelle pour un rapport.
N�anmoins, la structure devra toujours comporter des �l�ments qui en
permettent l'utilisation efficace.\index{Structure (rapport)}
Dans tous les cas le rapport devra �tre pagin� et le texte justifi�.

 Le rapport devra \emph{obligatoirement} contenir~:
 \begin{itemize}
 \item un r�sum�~; 
 \item une page titre~;
 \item une table des mati�res~;
 \item une introduction~;
 \item un d�veloppement~; 
 \item une conclusion~;
\item une bibliographie.
 \end{itemize}

\emph{Dans la plupart des cas}, on trouvera  �galement~:
\begin{itemize}
\item une liste de mots-cl�s~;  
\item une ou plusieurs annexes.
\end{itemize}

\emph{Dans certains cas}, il contiendra enfin~:
\begin{itemize}
\item un index~;
\item un glossaire.
\end{itemize}

Voyons les caract�ristiques de ces diff�rents �l�ments.

\subsection{Page de titre}

La page de titre doit permettre d'identifier le document. Pour cela, elle doit
comprendre~: \index{Page de garde}
\begin{itemize}
\item un titre~;
\item le type du document (par exemple \og{}Rapport de projet\fg{})~;
\item le nom du ou des auteurs du document~;
\item la date de parution (ou de remise du rapport) et �ventuellement le
  num�ro de version~;
\item le nom et le logo de l'organisme dont est issu l'auteur (par exemple
  \textsc{Ensta} Bretagne).
\end{itemize}

\subsection{R�sum�}

Le r�sum� fait la synth�se du projet en une page au maximum. Il
devra contenir une br�ve description du probl�me et des objectifs du projet,
mentionner les r�sultats les plus
importants et dresser une conclusion. Dans le r�sum� il faut surtout mettre en
�vidence les r�sultats car ce sont eux qui rendent mieux compte de votre
travail et donnent l'envie de lire votre rapport. 

Le r�sum� est souvent accompagn� de quelques mots-cl�s et il est, dans
certains cas, traduit en une ou plusieurs langues.\index{Resume@R�sum�}

\subsection{Table des mati�res}

Encore baptis�e \og{} sommaire \fg{}, la table des mati�res permet de
synth�tiser, en d�but de document, les diff�rents chapitres qui y sont
trait�s. Elle doit faire
r�f�rence � la pagination du document pour permettre au lecteur
d'acc�der directement � toute partie du document.\index{Table des
  matieres@Table des mati�res}\index{Sommaire} 

La table des mati�res se g�n�re automatiquement ; ne la cr�ez pas
manuellement, car vous pourriez oublier d'y apporter des
modifications.

\subsection{Introduction}

L'introduction doit permettre de situer le contexte de votre travail. En
particulier, elle doit~:\index{Introduction}
\begin{itemize}
\item pr�senter le contexte de l'�tude~;
\item d�finir le probl�me que le projet cherche � r�soudre~; 
\item d�finir le cadre du travail ; par exemple quelles ont �t� les
  contraintes, la m�thodologie choisie~;
\item annoncer le plan, qui n'est pas un ennonc� plat des diff�rentes parties,
  mais qui vise � d�finir l'orientation de la recherche, les hypoth�ses, le
  cheminement choisi pour tenter de r�pondre � la question.
\end{itemize}

\subsection{D�veloppement}

Le d�veloppement est la partie substantielle de votre document. Il doit �tre
structur� et d�coup� en plusieurs parties. Les encha�nements entre les parties
doivent �tre fluides (par exemple gr�ce � une courte introduction en d�but de
partie et un court bilan � la fin). Il peut s'appuyer sur des compl�ments
d'information (notes de bas de page, r�f�rences, annexes, glossaires ou listes
d'acronymes). N'h�sitez pas � illustrer vos propos par des figures, tables
ou �quations.

\subsubsection{Notes de bas de page}

Les notes de bas de page servent � apporter un compl�ment d'information
non essentiel. Le texte doit pouvoir �tre compris sans y faire appel.
Elles sont tr�s utiles lorsque l'on veut que le document puisse avoir
plusieurs niveaux de lecture.\index{Notes de bas de page}


\subsubsection{Tables}

De la m�me mani�re que les figures, les tables servent � illustrer certains
�l�ments du rapport. Elles doivent poss�der un titre et �tre
r�f�renc�es dans le corps du document. Il faut prendre garde � leur
lisibilit�. Un exemple \emph{� ne pas suivre} est pr�sent� table
\ref{tab:tab1} (table extraite du manuel de \LaTeX). Dans cet exemple, les
lignes (horizontales et verticales) sont 
trop nombreuses et nuisent � la lisibilit�, les alignements sont erratiques,
les unit�s ne sont pas correctement indiqu�es et les nombres sont repr�sent�s
avec des pr�cisions diff�rentes. Enfin, les titres de chaque colonne ne sont
pas apparents.\index{Tables}

\begin{table}[htbp]
  \centering
\begin{tabular}{||l|lr||} 
\hline
\hline
~~~~~gnats     & gram      & 013.65\euro \\ \cline{2-3}
          & each      & .01 \\ \hline
gnu       & stuffed   & 92.5 \\ \cline{1-1} \cline{3-3}
~~~emu       &           & 33.33 \\ \hline
armadillo & frozen    & 8.9887 \\ \hline
\end{tabular}
  \caption{Une table tr�s mal pr�sent�e}
  \label{tab:tab1}
\end{table}

La table \ref{tab:tab2} corrige tous les probl�mes de la table
\ref{tab:tab1}. Cette fois, la table est lisible et elle peut apporter un
compl�ment d'information au texte.

\begin{table}[htbp]
  \centering
\begin{tabular}{@{}llr@{}} 
  \toprule
  \multicolumn{2}{c}{\textbf{Item}} \\ 
  \cmidrule(r){1-2}
  \textbf{Animal} & \textbf{Description} & \textbf{Price (\euro)}\\ 
  \midrule
  Gnat  & per gram  & 13.65 \\
           & each      & 0.01 \\
      Gnu   & stuffed   & 92.50 \\
      Emu   & stuffed   & 33.33 \\
      Armadillo & frozen & 8.99 \\ 
      \bottomrule
\end{tabular}
  \caption{Une table correctement pr�sent�e}
  \label{tab:tab2}
\end{table}


\subsection{Conclusion}

La conclusion est une syth�se de votre travail et doit en faire un bilan
(critique) vis-�-vis des 
objectifs initiaux. 
\index{Conclusion}

\emph{Introduction et conclusion sont des parties essentielles d'un
  document. En les lisant, le lecteur doit pouvoir se faire une id�e
  pr�cise du contenu d�velopp� dans le corps du texte. Il est
  important d'y apporter le plus grand soin.}


\subsection{Annexes}

Les annexes regroupent les informations qui ne sont pas essentielles � la
compr�hension g�n�rale du rapport et dont la pr�sence dans le texte principal
nuirait � la fluidit� de la lecture. Les annexes peuvent par exemple
contenir~:\index{Annexes}
\begin{itemize}
\item des d�monstrations ou des calculs d�taill�s~;
\item des donn�es techniques (de mat�riel par exemple)~;
\item des donn�es brutes\footnote{uniquement si leur lecture apporte un
    compl�ment d'information utile au lecteur.}~;
\item des glossaires et index.
\end{itemize}


\subsection{Index et glossaire}

L'index et le glossaire sont plac�s dans les derni�res pages du
document. L'index permet de retrouver les termes cl�s du document par une
recherche alphab�tique~; il acc�l�re la recherche d'information dans les
rapports volumineux.\index{Index}

Le glossaire permet de regrouper la description de termes techniques et de
sigles � la fin du document. Il permet une meilleure compr�hension des
concepts dont l'explication n'est pas reprise en d�tail dans le texte.\index{Glossaire}


\subsection{R�gles de typographie}

Les usages de typographie diff�rent d'une langue � l'autre. Si le rapport est
�crit en anglais, vous devez alors respecter les usages
anglo--am�ricains~; consultez par exemple \autocite{typoUS} pour des
informations d�taill�es. \index{Typographie}

De m�me, dans tous les cas o� vous �crivez votre rapport en
fran�ais, il est important de respecter les r�gles de typographie fran�aise.
Dans ce cas,
la lecture de \autocite{andreTypo} vous donnera un tr�s bon aper�u des r�gles
� respecter et des �cueils � �viter.

\subsection{R�gles de pr�sentation de r�sultats num�riques}

Sauf dans le cas de valeurs dimensionnelles, les unit�s des r�sultats
num�riques doivent toujours �tre pr�cis�es. 

Il est �galement important d'�tre conscient du sens implicite donn� lors de la
pr�sentation de valeurs num�riques. En particulier, le nombre de chiffres
significatifs pr�sent� traduit la fid�lit� de la mesure. Les chiffres
significatifs excessifs doivent �tre supprim�s par arrondi.

\section{Outils d'aide � la production de documents}
\label{sec:outils}

Lors de votre scolarit� � l'\gls{enstab}, vous avez g�n�ralement le
choix des outils utilis�s pour la production de documents. 

\subsection{Traitement de texte et formateur de texte}

Un traitement de texte est un outil de mise en page de documents. Il est
g�n�ralement de type \gls{wysiwyg}\footnote{What You See Is What You Get}
et permet donc de voir l'aspect du document se modifier en continu. Dans cette
cat�gorie, les outils les plus connus sont \emph{Microsoft Word} et
\emph{LibreOffice}. Le formatage du texte est
obtenu par application de {\em styles}. Malheureusement, de nombreux
utilisateurs, insuffisamment form�s, n'utilisent pas ou utilisent mal
cette fonctionnalit�.

Le formateur de texte le plus connu est \LaTeX. Il proc�de en s�parant le fond
et la forme du document et poss�de des biblioth�ques de style qui permettent
de respecter les r�gles typographiques. Cet outil est g�n�ralement
non-\gls{wysiwyg} et requiert un certain investissement avant d'�tre
utilis� efficacement.


\subsection{Outils collaboratifs}

R�diger un rapport � plusieurs est toujours une op�ration
d�licate. L'utilisation d'outils collaboratifs permet de simplifier cette
phase, de manipuler plusieurs versions du document et de g�rer les conflits.
Plusieurs solutions sont disponibles~:
\begin{itemize}
\item utiliser le mode \og{}Modifications\fg{} de \emph{Microsoft Word} ou de
  \emph{LibreOffice}~; l'�change de fichiers et la gestion de version doivent
  alors �tre assur�s manuellement~;
\item utiliser un gestionnaire de version\footnote{Par exemple du \textsc{cvs}
  ou du \emph{subversion} h�berg� sur le serveur
  \url{https://gforge.ensta-bretagne.fr/gf} ou une gestion d�centralis�e avec
  \url{https://github.com}.} pour h�berger le code source
\LaTeX\footnote{Cette m�thode ne fonctionne que si le document est au format
  texte.}~;
\item h�berger le rapport sur un service \emph{cloud} tel que
  \emph{Google drive} ou \emph{Microsoft Office 365}.
\end{itemize}

\section{Conclusion}

Dans ce guide, les objectifs d'un rapport de projet ont
�t� pr�cis�s ainsi que la structure classique de son plan. Ensuite,
les r�gles de base pour sa r�daction ont �t� pr�sent�es, suivies d'un
rapide traitement des probl�matiques li�es aux outils de production de
documents. 



%%% Local Variables: 
%%% mode: latex
%%% TeX-master: "../guide"
%%% End: 


\appendix

\clearpage

\phantomsection
\addcontentsline{toc}{section}{Liste des tableaux}
\listoftables

\clearpage
\phantomsection
\label{sec:index}
\addcontentsline{toc}{section}{Index}
\printindex


\printglossaries


\clearpage
\phantomsection
\addcontentsline{toc}{section}{Bibliographie}

\printbibliography

\end{document}

%%% Local Variables: 
%%% mode: latex
%%% TeX-master: t
%%% End: 

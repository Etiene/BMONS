\chapter{Ingénierie des exigences}
\section{Approche Top-Down}
\label{sec:top-down}

\section{Approche Bottom-Up}

\section{Fonctions principales du système}

\chapter{Spécification fonctionnelle  3 axes}

\section{Raffinement FAST}
Le diagramme FAST regroupe les fonctions techniques globales définies dans les 
exigences ainsi que leur raffinements en sous fonctions et les solutions technique 
associées a celles-ci. Il a évolué au cours du projet en fonction des autres documents 
d'ingénierie système et des solutions techniques retenues. 

!!!  image du fast  !!!

\section{Spécification des données}
La spécification des données permet de mettre à jour les différentes grandeurs 
et unités intervenant dans notre système. Grâce à cela, nous savons exactement 
quel type donnée traiter et envoyer à l'apiculteur et/ou au serveur en fonction des évènements.

!!!  image de spécification des données  !!!

\section{Spécification des comportements}


\chapter{Architecture fonctionnelle}



\chapter{Ingénierie des exigences}
\section{Approche Top-Down}
\vspace{1.5cm}
Dans cette partie nous allons analyser notre système avec une approche Top-Down. Cela signifie que nous adopterons une démarche de conception descendante. Pour cela nous avons tracé le diagramme "bête à cornes", que l'on peut voir sur la figure \ref{fig:beteacorne}. Il permet de représenter graphiquement l'expression du besoin. Comme on peut le voir sur le diagramme, le système BMONS rend service aux apiculteurs en agissant sur une ou plusieurs ruches. Il a pour but d'aider la surveillance d'un rucher et d'avertir l'apiculteur en cas de problème.


\begin{figure}[h!]
\centering\includegraphics[scale=0.5]{BeteACornesBMONS.png}
\caption{\label{fig:beteacorne} Diagramme "bête à cornes" du système BMONS}
\end{figure}

Le diagramme pieuvre, \ref{fig:diagpieuvre1} et \ref{fig:diagpieuvre2}, nous permet ensuite de faire apparaître les fonctions principales du système. On peut aussi y retrouver les fonctions de services et de contraintes.

 
\begin{figure}[h!]
\centering\includegraphics[scale=0.6]{pieuvre1.png}
\caption{\label{fig:diagpieuvre1} Diagramme pieuvre du système BMONS}
\end{figure}

\begin{figure}[h!]
\centering\includegraphics[trim = 1cm 0.5cm 13cm 1cm,scale=0.6]{pieuvre2.png}
\caption{\label{fig:diagpieuvre2} Légende diagramme pieuvre du système BMONS}
\end{figure}

\clearpage

\section{Approche Bottom-Up}

\vspace{1.5cm}
Nous allons maintenant adopter la démarche inverse, mais néanmoins complémentaire, de l'approche Top-Down. Il s'agit de l'approche Bottom-Up. C'est une démarche de conception ascendante qui va nous permettre d'avoir une vision plus globale du système. On peut voir sur \ref{fig:exi1} et \ref{fig:exi2} les exigences issues de cette analyse.

 
\begin{figure}[h!]
\centering\includegraphics[trim = 1cm 8.7cm 1cm 1cm,scale=0.8]{Exigences_du_Projet_1.pdf}
\caption{\label{fig:exi1} Exigences issues de l'approche Bottom-Up (1/2)}
\end{figure}

 
\begin{figure}[h!]
\centering\includegraphics[trim = 1cm 2cm 1cm 1cm,scale=0.8]{Exigences_du_Projet_2.pdf}
\caption{\label{fig:exi2} Exigences issues de l'approche Bottom-Up (2/2)}
\end{figure}

\clearpage

\section{Fonctions métiers du système}
\vspace{1.5cm}

Après avoir réalisé une approche du point de vue "concepteur" du système, nous allons maintenant nous intéresser à la formulation des fonctions qu'un apiculteur souhaiterai avoir pour pouvoir suivre l'évolution de son rucher. Ces fonctions métiers ont été discutées avec notre client, monsieur Singhoff. Elles sont représentées sur les figures \ref{fig:exi1,exi2,exi3,exi4} en annexe dont la figure \ref{fig:exi1text} est un extrait.


\begin{figure}[h!]
\centering\includegraphics[trim = 1cm 2cm 1cm 1cm,scale=0.65]{Exigences1.pdf}
\caption{\label{fig:exi1text} Fonctions métiers du système (1/4)}
\end{figure}


\chapter{Spécification fonctionnelle 3 axes}

\section{Raffinement FAST}
\vspace{1.5cm}
Le diagramme FAST regroupe les fonctions principales, techniques et contraintes globales définies lors de l'établissement des exigences et après leurs discussion avec le client. Ainsi, certaines exigences que nous avons préalablement établies ont été acceptées mais d'autres ont écartées comme l'analyse des odeurs dans la ruche, jugée finalement trop compliquée à mettre en place et difficile à exploiter. D'autres exigences ont vu le jour après les échanges avec l'apiculteur, ce qui est venu enrichir et compléter notre tableau des exigences. Le raffinement des trois types de fonction en sous-fonction et les solutions techniques associées à celles-ci apparaissent également. Le diagramme FAST a évolué au cours du projet en fonction des autres documents d'ingénierie système et des solutions techniques retenues. 
On peut voir la version finale du FAST sur la figure \ref{fig:fast}

\begin{figure}[h!]
\centering\includegraphics[scale=0.8]{FAST_BMONS.pdf}
\caption{\label{fig:fast} Diagramme FAST du système BMONS}
\end{figure}

\clearpage

\section{Spécification des données}
\vspace{1.5cm}
La spécification des données permet de mettre à jour les différentes grandeurs et unités intervenant dans notre système. Grâce à cela, nous savons exactement quel type de données traiter et comment convertir les données numériques de sortie des capteurs en grandeurs physiques facilement compréhensible pour l'apiculteur. Il faudra ensuite envoyer ces données traitées au serveur qui les stockera dans une base de données. Cette étude a aussi permis d'établir les alertes qu'il va falloir prévoir afin d'avertir le propriétaire de l'état de son rucher. Ces dernières seront aussi stockées sur le serveur et viendront compléter l'historique de la ruche. Il est possible qu'une alerte soit envoyée après avoir effectué un recoupement d'informations. Par exemple, pour alerter l'apiculteur d'un essaimage en cours, il est nécessaire de détecter une agitation dans la ruche (chant de la reine) grâce au microphone et une chute importante de la masse grâce aux détecteurs de pression. 

Il existera deux types d'alertes: les alertes actives qui préconiserons à l'apiculteur d'agir directement sur la ruche (par exemple lorsque le développement de la grappe est excentrée par rapport aux cadres) et des alertes passives qui encourageront le client à se rendre sur le serveur pour consulter les derniers relevés.   

L'axe Data comportant la spécification des messages, des événements et des alarmes est fourni en annexe.
Le diagramme des données est décrit dans la figure \ref{fig:Fluxdonnees} 



\begin{figure}[h!]
\centering\includegraphics[scale=0.35]{diagFluxDonnees.jpg}
\caption{\label{fig:Fluxdonnees} Diagramme d'activité du flux de données}
\end{figure}

\clearpage

\section{Modèle de données}
\vspace{1.5cm}
Le modèle de données est une façon abstraite de représenter les données du système et de modéliser les informations contenues dans une base de données. Le modèle constitue des ensembles possédant un nom et des attributs nommés. Les relations sont faites via des clés primaires (id, dans notre cas) et des clés étrangères.  
La figure \ref{fig:UML} représente le modèle de données pour notre système. 



\begin{figure}[h!]
\centering\includegraphics[scale=0.55]{UML.png}
\caption{\label{fig:UML} Modèle de données pour le système BMONS}
\end{figure}

\clearpage

\section{Spécification des comportements}
\vspace{1.5cm}
Nous allons ici décrire le fonctionnement de notre système. Il est résumé dans la figure \ref{fig:sp_comp}.

\begin{figure}[h!]
\centering\includegraphics[scale=0.5]{specif_comp.pdf}
\caption{\label{fig:sp_comp} Diagramme de spécification des comportements}
\end{figure}

Les capteurs mesurent plusieurs paramètres internes à la ruche : l'humidité, la température et le poids de la ruche. Les données sont ensuite transmises au système embarqué. Dès que le système embarqué les reçoit, il traite les données, sélectionne celles qui sont valides et les envoi au serveur soit par le réseau sans fil (2G ou 3G), soit par le réseau téléphone si besoin. Une fois que le serveur reçoit les données, elles sont traitées et stockées dans une base de données sécurisée. Une représentation sous forme de graphiques permet une vision pratique et exploitable des informations par l'apiculteur.

Quand les mesures effectuées dépassent certain critères, par exemple, si la température est plus élevée que la température maximum pour la ruche, le serveur va générer un alerte qui sera envoyée à l'utilisateur, c'est-à-dire l'apiculteur, par SMS ou par e-mail. Par ailleurs, les apiculteurs peuvent consulter l’état de la ruche à distance afin de bien gérer la productivité de la ruche ou de limiter les situations problématiques. \\


La spécification des comportements se fait également grâce aux diagrammes de séquence. Ces diagrammes expliquent précisément comment les différentes parties du système interagissent dans le but d'exécuter une action. Ils détaillent l'ordre des actions ainsi que leur nature et leur durée d'exécution. Les 5 diagrammes de séquence qui ont été réalisés concernent l'action de récupération de la mémoire flash, figure \ref{fig:seq1}, l'analyse des information par le serveur, figure \ref{fig:seq2}, le service permettant à l'apiculteur d'avertir le serveur d'une intervention sur une ruche, figure \ref{fig:seq3}, la fonction permettant la configuration des alertes, figure \ref{fig:seq4} et l'action de connexion d'un apiculteur à son compte, figure \ref{fig:seq5}.
\begin{figure}[h!]
\centering\includegraphics[scale=0.7]{recupererMemoireFlash.jpg}
\caption{\label{fig:seq1} Diagramme de récupération de la mémoire flash}
\end{figure}\\

\newpage

La figure \ref{fig:seq1} se lit de la façon suivante les quatre blocs en haut représentent les acteurs de l'action recupererMemoireFlash. Ces acteurs interagissent par l'intermédiaire de requête ou d'appel de fonction représentés par les flèches. Le tout se déroule selon un axe temporel symbolisé par les lignes verticales sous les acteurs.
\begin{figure}[h!]
\centering\includegraphics[scale=0.4]{analyserInformation.jpg}
\caption{\label{fig:seq2} Diagramme d'analyse de l'information}
\end{figure}
\begin{figure}[h!]
\centering\includegraphics[scale=0.4]{avertirOperationRuche.jpg}
\caption{\label{fig:seq3} Diagramme d'avertissement d'opération sur la ruche}
\end{figure}
\begin{figure}[h!]
\centering\includegraphics[scale=0.4]{configurerAlerte.jpg}
\caption{\label{fig:seq4} Diagramme de configuration des alertes}
\end{figure}
\begin{figure}[h!]
\centering\includegraphics[scale=0.7]{IdentificationCompte.jpg}
\caption{\label{fig:seq5} Diagramme d'identification compte}
\end{figure}
\pagebreak

\chapter{Architecture fonctionnelle}

L'étude de la spécification fonctionnelle trois axes a permis d'établir l'architecture fonctionnelle du système qui est représentée sur la figure \ref{fig:anaFonc}.
Ce schéma résume les interactions entre chaque partie: Bee Monitor qui regroupe l'ensemble des capteurs, la carte Arduino ainsi que la carte SSD pour l'enregistrement local des données, le module de transmission et le système d'alimentation rendant notre projet autonome en énergie et le serveur. Chaque acteur interagissant avec le système est également représenté: Les abeilles/ruche, l'apiculteur et l'environnement.   
  

\begin{figure}[h]
\centering\includegraphics[scale=0.5]{analyseFonctionelle1.pdf}
\caption{\label{fig:anaFonc} Architecture fonctionnelle}
\end{figure}    


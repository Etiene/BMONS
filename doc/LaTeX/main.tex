\documentclass[12pt,fleqn]{book} %taille de la police par défaut, et équations jusitifées à gauche
\usepackage[top=3cm,bottom=3cm,left=3.2cm,right=3.2cm,headsep=10pt,a4paper]{geometry} % marges
\usepackage{xcolor}
\definecolor{enstabGreen}{HTML}{C8D200} 	%vert  	#c8d200 
\definecolor{enstabLightGreen}{HTML}{E9ED99} 	%vert  	#c8d200 
\definecolor{enstabLightBlue}{HTML}{009EE0} %bleu clair 	#009ee0
\definecolor{enstabVeryLightBlue}{HTML}{99D8F3} %bleu clair 	#009ee0
\definecolor{enstabDarkBlue}{HTML}{005C8F}	%bleu foncé 	#005c8f
\definecolor{enstabDarkGrey}{HTML}{333333}	%gris fort 	#333333
\definecolor{enstabLightGrey}{RGB}{48,48,48}	%gris fort 	#333333
\definecolor{enstabParme}{HTML}{8878B2}		%parme 	#8878b2
\definecolor{enstabOrange}{HTML}{F18E00} 	%orange 	#f18e00
\usepackage[colorlinks=true,
        urlcolor=enstabLightBlue,
        anchorcolor=enstabDarkBlue,
        linkcolor=enstabDarkBlue,
        citecolor=enstabDarkGrey,
        pdfauthor={Johan B. C. Engelen},
        pdfkeywords={SVG; LaTeX; Inkscape},
        pdftitle={How to include an SVG image in LaTeX},
        pdfsubject={Describes how to include an SVG image easily in LaTeX using Inkscape}] {hyperref}
\usepackage{url}
\usepackage[utf8]{inputenc} % lettres accentuées
\usepackage[T1]{fontenc}    % Use 8-bit encoding that has 256 glyphs
\usepackage[frenchb]{babel} % Pour le français
\usepackage{cclicenses}     % Licences CC
\usepackage{epigraph}
\usepackage{eso-pic}        % pour une image en fond, page de titre
\usepackage{graphicx}       % Pour inclure des images
\graphicspath{{images/}}    % Où sont les images ?

\usepackage{listings}      % Pour coloriser les codes que vous insérez
\lstset{ %
  backgroundcolor=\color{white},   % choose the background color; you must add \usepackage{color} or 
  basicstyle=\footnotesize\ttfamily,        % the size of the fonts that are used for the code
  breakatwhitespace=false,         % sets if automatic breaks should only happen at whitespace
  breaklines=true,                 % sets automatic line breaking
  captionpos=b,                    % sets the caption-position to bottom
  commentstyle=\color{enstabOrange},    % comment style
  deletekeywords={...},            % if you want to delete keywords from the given language
  escapeinside={\%*}{*)},          % if you want to add LaTeX within your code
  extendedchars=true,              % lets you use non-ASCII characters; for 8-bits encodings only, does not work with UTF-8
  %frame=single,                    % adds a frame around the code
  keepspaces=true,                 % keeps spaces in text, useful for keeping indentation of code (possibly needs columns=flexible)
  keywordstyle=\color{enstabDarkBlue},       % keyword style
  %language=Octave,                 % the language of the code
  morekeywords={*,...},            % if you want to add more keywords to the set
  numbers=left,                    % where to put the line-numbers; possible values are (none, left, right)
  numbersep=8pt,                   % how far the line-numbers are from the code
  numberstyle=\tiny\color{enstabDarkGrey}, % the style that is used for the line-numbers
  rulecolor=\color{black},         % if not set, the frame-color may be changed on line-breaks within not-black text (e.g. comments (green here))
  showspaces=false,                % show spaces everywhere adding particular underscores; it overrides 'showstringspaces'
  showstringspaces=false,          % underline spaces within strings only
  showtabs=false,                  % show tabs within strings adding particular underscores
  stepnumber=5,                    % the step between two line-numbers. If it's 1, each line will be numbered
  stringstyle=\color{enstabParme},     % string literal style
  tabsize=2,                       % sets default tabsize to 2 spaces
  title=\lstname                   % show the filename of files included with \lstinputlisting; also try caption instead of title
}





\usepackage{booktabs}       % pour de jolis tableaux
%\usepackage{fancyhdr}       % pour des entêtes et pieds de pages améliorés.
\usepackage{makeidx}        % requis pour faire les index
\usepackage{glossaries} %requis pour faire le glossaire
     % Ce fichier contient tous les packages nécessaires à la compilation
\makeindex           % donne l'ordre de créer l'index
%\includegraphics[scale=1]{image}
\newacronym{IS}{IS}{Ingénierie Système}
\newacronym{WBS}{WBS}{Work Breakdown Structure}					 % contient les entrées du glossaire
\makeglossaries      % donne l'ordre de créer le glossaire

\begin{document}
\renewcommand{\contentsname}{Sommaire}                % des jolis noms pour la table des matières
\renewcommand{\bibname}{Références bibliographiques}  % des jolis noms pour les sections bibliographiques
\renewcommand{\glossaryname}{Glossaire}               % et glossaire


%----------------------------------------------------------------------------------------
%	 PAGE DE TITRE
%-----------------------------	-----------------------------------------------------------

\begingroup
\thispagestyle{empty}
\AddToShipoutPicture*{\put(6,5){\includegraphics[scale=1]{FondTitreSPID}}} % Image background
\begin{center}
\vspace*{2cm}
{\Huge \textsc{\textbf{Rapport d'avancement}}}\\
\vspace*{2cm}
{\Huge \textbf{BMONS}}\par % ACRONYME du projet
\vspace*{2cm}
{\huge Beehive Monitoring System}\par % Intitulé du projet
\end{center}
\vspace*{4cm}

\textbf{\huge Rédigé par :} 

\begin{center}
{
\huge
Alice Danckaers\\
Benoît Raymond\\
Etiene Dalcol\\
Nicolas Van-Nhân Nguyen\\
Tao Zheng\\
Armand Sellier\\
}
\end{center}

\vspace*{1cm}

{\huge \textbf{Sous la direction de :}}\\
\begin{center}
{\huge
Olivier Reynet\\
}
\end{center}
\endgroup


%----------------------------------------------------------------------------------------
%	COPYRIGHT PAGE
%----------------------------------------------------------------------------------------
\newpage
~\vfill
\thispagestyle{empty}

\noindent \bysa 2014 Alice Danckaers, Benoît Raymond, Etiene Dalcol, Nicolas Van-Nhân Nguyen, Tao Zheng and Armand Sellier.\\\\ % Copyright notice

%\noindent \textsc{Published by Publisher}\\ % Publisher

%\noindent \textsc{book-website.com}\\ % URL

\noindent Licensed under the Creative Commons Attribution-ShareAlike 4.0 International Public License.\\ % License information

\noindent \textit{Première impression, décembre 2014} % Printing/edition date

%----------------------------------------------------------------------------------------
%	SOMMAIRE
%----------------------------------------------------------------------------------------
\tableofcontents  % Imprime le sommaire
\cleardoublepage  % pour commencer sur une page impaire

%----------------------------------------------------------------------------------------
%	Préambules
%----------------------------------------------------------------------------------------
\frontmatter      % La partie non numérotée préalable au document principal


\chapter{Remerciements}
\epigraph{La gratitude est non seulement la plus grande des vertus, mais c'est également la mère de tous les autres.}{Emil Cioran}

Je tiens à remercier tous les contributeurs de \LaTeX qui nous permettent aujourd'hui de produire des documents de qualité professionnelle sans avoir à se préoccuper de son apparence : des livres, des articles, des mémentos dans presque toutes les langues, mais aussi de la musique et des dessins. Ce logiciel ne connaît pas de limites.
 

%\chapter{Préambule}
\epigraph{Le chemin est long du projet à la chose.}{Molière}

\section{Comment compiler ce document ?}

Un document \LaTeX peut se compiler au travers d'un IDE (TexSutdio, TeXMaker par exemple).
Le répertoire de ce document contient également un Makefile qui permet de compiler simplement en ligne de commande. 
La fabrication du glossaire et de l'index est prise en charge par ce Makefile.
Pour l'utiliser, il suffit d'ouvrir un terminal, de se placer dans le répertoire du document puis d'invoquer la commande \texttt{make}. 


Voici les différentes cibles disponibles pour ce Makefile :
\begin{verbatim}
make                - contruit le document
make all            - contruit le document
make index          - contruit l'index
make glossaire      - contruit le glossaire
make bib            - contruit la bibliographie
make pdf            - contruit le document PDF
make clean          - supprime les fichiers LaTeX intermédiaires
make clean-all      - supprime tous les fichiers générés par la compilation
make help           - cette information
\end{verbatim}


\section{Références internes}

Les références internes sont des renvois vers des figures, des tableaux ou des sections du rapport.
\LaTeX introduit un mécanisme simple pour établir ce genre de référence, via les commandes \texttt{\textbackslash label} et \texttt{\textbackslash ref}. 
La première sert à définir une ancre dans le document, la seconde à la citer.
Voici par exemple une référence interne vers la section intitulée \textit{Approche Top-Down} (cf. section  \ref{sec:top-down}). Ce renvoi est le résultat de la commande \texttt{\textbackslash ref\{sec:top-down\}}. Si vous vous rendez dans le corps de cette section, vous y trouverez le label en question \texttt{\textbackslash label\{sec:top-down\}}. 

\subsection{Tableaux et figures}
Les figures  et les tableaux  sont référencés de la même la manière (cf. figure \ref{fig:gomboc} et tableau \ref{tab:exemple}). \index{Table} \index{Figure}

\begin{table}[h]
\centering
\begin{tabular}{lll}
\toprule
\textbf{Algorithmes} & \textbf{Performance (s)} & \textbf{Gain (dB)}\\
\midrule
Algorithme 1 & 0.0003262 & 0.562 \\
Algorithme 2 & 0.0015681 & 0.910 \\
Algorithme 3 & 0.0009271 & 0.296 \\
\bottomrule
\end{tabular}
\caption{\label{tab:exemple}Performances et gains des algorithmes envisagés.}
\end{table}

\begin{figure}[h]
\centering\includegraphics[scale=0.15]{Gomboc.jpg}
\caption{\label{fig:gomboc}Gömböc : un objet homogène tridimensionnel mono-monostatique. (source : Wikipedia)}
\end{figure}



\subsection{Codes}
Si vous souhaitez insérer du code dans votre rapport, invoquez les commandes : \\
 \texttt{\textbackslash lstset\{language=python\}}\\
 \texttt{\textbackslash lstinputlisting[caption=\{Titre du listing\}, label=\{lst:code\}]\{./code/code.py\}}


La première commande sélectionne le langage, pour que les mots clés de celui-ci soient correctement détectés et mis en valeur. 
La seconde commande permet d'insérer le code contenu dans le fichier code.py qui se trouve dans le sous-répertoire code.
Pour faire référence au code, il suffit de sélectionner le label du listing \ref{lst:prime},  comme pour les figures et les tableaux.

\lstset{language=python}
\lstinputlisting[caption={Titre du listing}, label={lst:prime}]{./code/primegen.py}


\subsection{Index et glossaire}

Pour insérer des entrées dans l'index, il suffit de déclarer des mots via la commande \texttt{\textbackslash index\{Fabrication d'un index\}} comme suit\footnote{Allez donc voir l'index \ref{sec:index} à la fin du document  !}. \index{Fabrication d'un index}

Pour utiliser le glossaire, il faut définir les termes dans le fichier \texttt{glossaire.tex} en utilisant la commande \texttt{\textbackslash newacronym\{label\}\{abbréviation\}\{Signification\}}. 
Puis,  \texttt{\textbackslash gls\{label\}} permet de les utiliser dans le document. 


Par exemple, les UVs 3.4 et 4.4 sont une initiation à l'\gls{IS}. 
Un concept de gestion de projet souvent mal connu est le \gls{WBS}.


\section{Références bibliographiques}

Les références bibliographiques sont des documents numériques, des livres, des articles, des images ou des vidéos qui ne sont pas présents dans le rapport. 
\LaTeX propose un mécanisme simple de citation.
Pour plus de détails, vous pouvez consulter les références suivantes \cite{maguis2010redigez,desgraupes2003latex,bitouze2010latex} qui sont présentent à la médiathèque de l'ENSTA Bretagne, ou celle-ci directement sur le web \cite{openclassroomLaTeX}.  

Pour citer des documents, il suffit d'appeler la commande \texttt{\textbackslash cite\{key\}} en choisissant la clé qui identifie le document, comme suit : \cite{lamport1985i1}. 
Cette clé de citation est celle qui référence l'ouvrage dans le fichier de bibliographie intitulé   \texttt{bibliographie.bib}.
Ce fichier d'exemple contient tous les types de documents dont vous aurez besoin : livre, article de journal, références web,  rapport\dots 
Une fois insérée et compilée, la citation devient un lien dans le fichier pdf, redirigeant ainsi directement vers le détail de l'ouvrage cité dans la bibliographie située à la fin du document.
 


\mainmatter       % La partie principale du document

%----------------------------------------------------------------------------------------
%	PART I 
%----------------------------------------------------------------------------------------
\part{Introduction au projet}
\chapter{Formulation initiale du projet}



\section{Contexte}

BeeHive Monitoring System (BMONS) est un projet qui a pour but d'aider les apiculteurs. Il s'agit de leur proposer un système de surveillance et de détection peu onéreux afin de prodiguer les meilleurs soins au meilleur moment aux ruches qui en ont besoin et d'éviter les vols.

En effet, les abeilles sont vitales à l'équilibre écologique. Einstein avait même dit: " Si l’abeille disparaît, l’humanité en a pour quatre ans à vivre ". Sans elles 84 \% des espèces végétales cultivées pour l'alimentation disparaitraient. Or les abeilles sauvages sont aujourd'hui rares et l'espèce ne survivrai pas sans l'aide des apiculteurs. Ainsi le travail de ces derniers est crucial non seulement pour assurer la production de miel mais aussi pour la sauvegarde de l'environnement. Cependant, ces dernières années, les apiculteurs ont été confrontés à de nombreux problèmes et nous sommes aujourd'hui face à une diminution du nombre d'abeilles telle que la production annuelle européenne de miel est quatre fois moindre que celle qu'il y a vingt ans. 

Pour aider à la résolution de ce problème, nous voulons donc créer un système capable d'aider l'apiculteur dans son travail et de ce fait combattre la disparition des abeilles. 

\section{Expression initiale du besoin}

Après avoir discuté avec plusieurs apiculteurs, nous avons pu identifier leurs besoins et déterminer de quelle manière nous pouvons les aider. Ainsi l'objectif de ce système est tout d'abord de donner accès à l'apiculteur à des informations clés sur la ruche sans que celui-ci n'ai à se déplacer, ni à ouvrir les ruches. En effet l'ouverture de la ruche perturbe les abeilles et elle n'est pas possible en hiver à cause des températures trop basses. De plus les ruches sont souvent disposées dans des ruchers éloignés les uns des autres, ce qui complique le travail de l'apiculteur. Les informations nécessaires seraient : la température dans et en dehors de la ruche, le poids, l'humidité et les sons de la ruche. Mais le système devra aussi alerter l'apiculteur quand la sécurité de la ruche est compromise, pour permettre une action rapide destinée à sauver la colonie.

Le système BMONS est donc composé de deux parties distinctes. La première consiste en un élément embarqué dans la ruche qui consomme un minimum d'énergie et qui mesure les paramètres clés. Les données de cet élément embarqué sont transmises via un transmetteur sans fils à un serveur qui constitue la deuxième partie du système. Il donne accès à l'apiculteur aux différentes mesures effectuées dans et autour des ruches. Il envoie également des alertes de sécurités à l'apiculteur si besoin. 


\chapter{État de l'art}

En effectuant nos recherches sur le sujet nous avons trouvé beaucoup d'informations sur les abeilles et le travail des apiculteurs en général, ainsi que des systèmes "maison" développés par des particuliers pour surveiller un peu mieux leurs ruches. Cependant nous avons également découvert l'existence de quatre projets similaires au notre: trois projets en cours ayant une approche OpenSource et un projet commercial déjà développé. Ce dernier appartient à la société anglaise Arnia. Ce système est décrit [CF SITE] comme permettant à l'utilisateur d'avoir des informations sur une ou plusieurs ruches telles que la température, l'humidité et l'intensité acoustique dans la ruche ainsi que la température du couvain. Les apiculteurs peuvent ensuite visualiser ces informations sur une partie sécurisée du site internet d'arnia. Ils peuvent également comparer les informations et évolution d'une ou plusieurs ruches, comme on peut le voir sur la figure XXX.

IMAGE

L'un des projets OpenSource est développé par Ken Meyer sur le site hackaday [CF SITE] et consiste à mesurer la température, l'humidité et le poids d'une ruche. Ce projet est encore en développement et plusieurs prototypes ont déjà été testés.

Il existe également un autre projet OpenSource sur le sujet. Il s'agit de Bzzz[CF SITE], développé par le Fablab de Lannion. Ce système propose une supervision de la température intérieure, de la luminosité extérieure et la masse d'une seule ruche via un envoi de données périodique par SMS et par visualisation des données sur un portail en ligne. L'utilisateur pourra également configurer des alertes via le portail.

Enfin le dernier système existant que nous avons trouvé a été développé conjointement par le Fablab de Barcelone et Open Tech Collaborative, Denver, USA [CF SITE]. Ce projet OpenSource, appelé Open Source BeeHive, ne s'adapte pas aux ruches classiques mais propose une architecture simple qui permet de construire sa propre ruche entièrement, comme on peut le voir sur la figure XXX.

IMAGE

Ensuite un kit de capteurs à installer permet de mesurer la température, l'humidité, l'intensité acoustique et le nombre d'abeilles via un capteur infrarouge. Les données seront ensuite visibles de tous sur la plateforme Smart Citizen.

Nous n'avons pas détaillé ici tous les projets que nous avons trouvés du fait de leur grand nombre. Cependant nous nous sommes intéressés à ceux qui présentaient un intérêt pour le système que nous voulons développer. 

%----------------------------------------------------------------------------------------
%	PART II 
%----------------------------------------------------------------------------------------
\part{Dossier fonctionnel}
\chapter{Ingénierie des exigences}
\section{Approche Top-Down}
\label{sec:top-down}

\section{Approche Bottom-Up}

\section{Fonctions principales du système}

Les fonctions principale, services et contraintes du système sont regroupées dans ce diagramme pieuvre, figure \ref{fig:diagpieuvre}. Les buts et les contraintes imposées au système y sont également représentés. 
 
\begin{figure}[h]
\centering\includegraphics[scale=0.7]{diagpieuvre.pdf}
\caption{\label{fig:diagpieuvre} Diagramme pieuvre de système BMONS}
\end{figure}


\chapter{Spécification fonctionnelle  3 axes}

\section{Raffinement FAST}
Le diagramme FAST regroupe les fonctions techniques globales définies dans les 
exigences ainsi que leur raffinements en sous fonctions et les solutions technique 
associées a celles-ci. Il a évolué au cours du projet en fonction des autres documents 
d'ingénierie système et des solutions techniques retenues. On peut voir la version finale du FAST sur la figure \ref{fig:fast}

\clearpage

\begin{figure}[h!]
\centering\includegraphics[scale=0.7]{FAST_BMONS.pdf}
\caption{\label{fig:fast} Diagramme FAST du système BMONS}
\end{figure}


\section{Spécification des données}
La spécification des données permet de mettre à jour les différentes grandeurs 
et unités intervenant dans notre système. Grâce à cela, nous savons exactement 
quel type donnée traiter et envoyer à l'apiculteur et/ou au serveur en fonction des évènements. Cette étude a aussi permis d'établir les alertes qu'il va falloir prévoir afin d'avertir le propriétaire de l'état de son rucher. 

!!!  image de spécification des données  !!!

\section{Spécification des comportements}

Nous allons ici décrire le fonctionnement de notre système. Il est résumé dans la figure \ref{fig:sp_comp}.

\begin{figure}[h!]
\centering\includegraphics[scale=0.7]{specif_comp.pdf}
\caption{\label{fig:sp_comp} Diagramme de spécification des comportements}
\end{figure}

Les capteurs mesurent plusieurs critères : l'humidité, la température et le poids de la ruche. Les données sont ensuite transmises au système embarqué. Dès que le système embarqué les reçoit, il traite les données, sélectionne celles qui sont valides et les envoi au serveur soit par le réseau sans fil, soit par le réseau téléphone si besoin. Une fois que le serveur reçoit les données, elles sont traitées et stockées dans une base de données sécurisée. Une représentation sous forme de graphiques permet une vision pratique et exploitable des informations par l'apiculteur.

Quand les mesures effectuées dépassent certain critères. Par exemple, si la température est plus hausse que la température maximum pour la ruche, le serveur va générer un alerte qui sera envoyée à l'utilisateur, c’est-à-dire l'apiculteur, par SMS ou par e-mail. Par ailleurs, les apiculteurs peuvent consulter l’état de la ruche à distance afin de bien gérer la productivité de la ruche ou de limiter les situations problématiques. 

\pagebreak

\chapter{Architecture fonctionnelle}




%----------------------------------------------------------------------------------------
%	PART III 
%----------------------------------------------------------------------------------------
\part{Implémentation}

\chapter{Architecture physique}

\section{Architecture physique matérielle}
\vspace{1.5cm}
La solution technique que nous avons choisie se présente sous la forme d'un cadre en bois léger (type contreplaqué) dans lequel les capteurs sont incrustés. On peut voir sur la vue du dessus du cadre en coupe,figure \ref{fig:face}, la position des différents capteurs.Sur deux côtés opposés il y aura 4 capteurs de pression et 3 capteurs de température, voir figure \ref{fig:cote1}. Puis sur un autre côté le capteur d'humidité, le microphone et la sortie des fils, voir figure \ref{fig:cote2}. Enfin sur le dernier côté il y aura uniquement un microphone. Les capteurs seront placés entre deux épaisseurs de bois préalablement travaillées et dépasseront si nécessaire du cadre. Les fils de connexion seront rassemblés et sortiront à un seul endroit du cadre. Ils seront placés dans une gaine protectrice et se connecteront au boîtier extérieur. Ce dernier sera placé à côté de la ruche et comprendra la carte arduino et son shield 3G, la carte SD et la batterie. 

\begin{figure}[h!]
\centering\includegraphics[trim= 9cm 6cm 8cm 2cm,scale=0.8]{cadre_cote1.png}
\caption{\label{fig:cote1} Schéma du cadre de mesure. Coupe vue de coté}
\end{figure}

\begin{figure}[h!]
\centering\includegraphics[trim= 5cm 5cm 5cm 5cm,scale=0.8]{cadrecote2.png}
\caption{\label{fig:cote2} Schéma du cadre de mesure. Coupe vue de coté}
\end{figure}

\begin{figure}[h!]
\centering\includegraphics[trim= 1cm 1cm 2cm 3cm,scale=0.8]{cadre_face.png}
\caption{\label{fig:face} Schéma du cadre de mesure. Coupe vue du dessus}
\end{figure}


\clearpage

\section{Architecture physique logicielle}
\vspace{1.5cm}
Dans cette partie nous allons analyser l'architecture physique côté logiciel. Elle est divisée en deux parties: Serveur et Arduino. La partie serveur comprend tout ce qui est lié au développement du site et les scripts qui contrôlent les logiciels, les backups et l'obtention de données. La partie arduino comprend tous les scripts de contrôle des capteurs et manipulation des données. Ceci est résumé dans le diagramme \ref{SDP}. 


\begin{figure}[h!]
\centering\includegraphics[scale=0.55]{WBS.png}
\caption{\label{fig:SDP} Architecture physique logicielle du système BMONS}
\end{figure}

\newpage

Au niveau de la carte Arduino, nous avons commencé à prendre en main les capteurs dès que nous les avions reçus notamment les capteurs de température et de pression. Nous avons commencé à coder un programme qui permettrai de récupérer la température en degrés.(voir figure \ref{code}). Pour la pression, nous récupérons une certaine valeur de résistance. A partir de cela, nous comptons effectuer un étalonnage afin d'en tirer une information sur la masse de la ruche ou des hausses selon la saison et l'envie de l'apiculteur. Cette étalonnage se fera en fonction des informations présentes sur la documentation constructeur du capteur de pression.

\begin{figure}[h!]
\centering\includegraphics[scale=0.45]{codeArduino.png}
\caption{\label{fig:code} Programmation de la carte Arduino}
\end{figure}  

\clearpage
%\input{Interfaces}
%\chapter{Structure de découpage du projet}

SDP en français, ou WBS Work Breakdown Structure en anglais 
\chapter{Tests}

Après avoir pris en main nos capteurs, nous avons prévu de tester leur validité en préparant une série de tests pour chacun d'entre eux. Cela nous permettra notamment de vérifier leur précision et les conséquences d'un dépôt de miel ou de propolis à leur surface. Ainsi, les résultats de ses test pourra affecter leur position dans la ruche. 

\section{Test de résistance à la cire d'un capteur de température}

\noindent \underline{Matériel}:  
3 capteurs de température  
Cire d'abeilles  
1 support\newline  

\noindent \underline{Préparation:} \newline 
Fixer les capteurs de température sur le support espacés de manière régulière sur le capteur de gauche appliquer 0.5 cm de cire sur le capteur de droite appliquer 1 cm de cire le capteur central sera le témoin \newline

\noindent \underline{Test 1:} \newline 
Mettre le support au frigo et surveiller l'évolution de la température\newline   
Relever le temps nécessaire pour arriver à la même température que le témoin pour les 2 capteurs tests  

\noindent \underline{Test 2:}  \newline 
Directement après le test 1 enlever le support du frigo et surveiller l'évolution de la température\newline   
Relever le temps nécessaire pour arriver à la même température que le témoin pour les 2 capteurs tests

\noindent \underline{Test 3:}  \newline 
Mettre le support dans un "four" à 30 degrés et surveiller l'évolution de la température\newline   
Relever le temps nécessaire pour arriver à la même température que le témoin pour les 2 capteurs tests  
  
\noindent \underline{Test 4:}  \newline 
Directement après le test 3 enlever le support du "four" et surveiller l'évolution de la température\newline   
Relever le temps nécessaire pour arriver à la même température que le témoin pour les 2 capteurs tests


\newpage

\section{Test Capteur d'humidité} 

\noindent \underline{Matériel}: 
1 capteur d'humidité des sachets déshydratants\newline  
2 boites/sachets hermétiques cotons et eau \newline 

\noindent \underline{Préparation:} \newline  
Commencer et finir le cycle avec un capteur propre\newline  

\noindent \underline{Etape 0:}  

\noindent Relever le niveaux d'humidité d'origine  

\noindent \underline{Etape 1:} \newline  
Mettre le support dans réceptacle à 100\% d'humidité\newline  
Surveiller l'évolution\newline  
Relever les courbes

\noindent \underline{Etape 2:} 

\noindent Mettre le support dans un réceptacle à 0\% d'humidité 

\noindent \underline{Etape 3:}  \newline
Répéter les  étapes 1 et 2 cinq fois avec différentes épaisseurs


%----------------------------------------------------------------------------------------
%	PART IV 
%----------------------------------------------------------------------------------------
%\part{Intégration et validation}
%\input{Integration}
%\input{Validation}

%----------------------------------------------------------------------------------------
%	PART V 
%----------------------------------------------------------------------------------------
\part{Organisation}
\chapter{Méthodes de travail}

% Méthodes de travail
% Organisation temporelle, spatiale, humaine 
% interactions des membres de l’équipe projet
% interactions avec les encadrants
% interactions avec les tiers

\section{Communication}
\vspace{1.0cm}
Tout au long du projet notre méthode de travail a changé. Au fur et à mesure que les séances s’enchaînaient, et en prenant en compte les conseils qui nous ont été donnés (notamment à propos du carnet de bord et des objectifs à court terme) notre méthode de travail a tendu vers la suivante : \\ \\
De manière générale, toute l'équipe du projet BMONS travaille dans la même salle pour faciliter la communication entre les membres du groupe. Une séance de travail commence par l'ouverture personnelle des mails de chacun, puis le groupe se réunit pour définir les objectifs de la matinée, ensuite chacun choisit la partie sur laquelle il va avancer. Le travail se fait en général seul ou en binôme et des points d'avancement sont faits à l'oral tout au long de la séance. Parfois des tâches comme la prise en main d'un logiciel ou la compréhension d'un diagramme sont faites en dehors des séances, mais la majeure partie du travail s'effectue lors du temps alloué au projet. \\ \\
Les réunions avec les encadrants et les intervenants extérieurs se déroulent dans des salles de l'ENSTA Bretagne équipées d'un vidéo projecteur et en présence de la totalité de l'équipe BMONS. Ces séances sont organisées à l'avance, les points sur lesquels des précisions sont nécessaires sont mis en avant au préalable et des questions précises sont préparées. Cela permet de guider la réunion et de ne pas perdre de temps sur des points déjà vus. Chacun a son rôle lors de ces réunions, la prise de notes, le dialogue avec l'intervenant et la rédaction du compte rendu sont ainsi facilités.

\section{Outils pour les échanges}
\vspace{1.5cm}
% Quels sont les outils qui nous permettent de travailler ensemble ?

Les outils qui nous ont permis de travailler ensemble et de partager nos fichiers ont également changés au cours du temps. 
Avant les premiers ateliers techniques, particulièrement celui sur github, nous partagions nos résultats sur le oneDrive d'office 365. Cet outil est efficace pour partager des documents, mais n'est pas optimisé pour le travail simultané de plusieurs membre du groupe sur un même document. \\
Nous avons donc utilisé github (figure \ref{fig:InterfaceGit}), malgré une prise en main un peu longue pour ceux d'entre nous qui n'ont pas assisté à l'atelier, github s'est révélé être un outil optimisé pour le travail que nous faisons. Le journal de bord, et les objectifs à cours termes en particulier, sont regroupés dans un wiki (figure \ref{fig:ExempleWiki}) que chacun peu consulter, les liens utiles au projet y sont également répertoriés. Notre projet étant open source, la totalité de ces informations est accessible par tous ceux qui souhaitent s'informer sur l'avancement du projet, ainsi nous avons reçus un mail de la part d'une de ces personnes nous invitant à suivre un projet similaire.

\begin{figure}[h!]
\centering\includegraphics[scale=0.5]{InterfaceGit.png}
\caption{\label{fig:InterfaceGit} interface de travail gitHub}
\end{figure}

\begin{figure}[h!]
\centering\includegraphics[scale=0.55]{ExempleCarnetDeBord.png}
\caption{\label{fig:ExempleWiki} Exemple d'utilisation du wiki de gitHub}
\end{figure}

\chapter{Répartition des tâches}

\section{Structure de découpage du projet}
\vspace{1.0cm}
Structure de découpage du projet, ou Work Breakdown Structure (WBS) en anglais, est un diagramme hiérarchique, axé sur les tâches et activités que les membres de l'équipe doivent exécuter pour atteindre les objectifs du projet.

Dans cette partie nous allons analyser nos tâches. Elles sont divisés en deux parties: Serveur et Arduino. La partie serveur comprend tout ce qui est lié au développement du site ainsi que les scripts qui contrôlent les logiciels, les backups et l'obtention de données. La partie arduino comprend tous les capteurs et modules, la gestion de l'énergie et les scripts de contrôle des capteurs et manipulation des données.  
\begin{figure}[h!]
\centering\includegraphics[scale=0.40]{WBS2.png}
\caption{\label{fig:SDP} Structure de découpage du projet du système BMONS}
\end{figure}

\clearpage

\section{Diagramme de Gantt}
\vspace{1.5cm}
Le diagramme de Gantt est un outil utilisé en gestion de projet permettant de visualiser dans le temps les diverses tâches composant un projet. Il s'agit de représenter graphiquement l'avancement du projet en fonction des séances. Les tâches du projet BMONS sont divisées en deux catégories, Serveur et Arduino, comme dans le structure de découpage du projet. Ce qui permet à l'équipe de se séparer en deux et de partager les tâches en travaillant de façon concomitante.
\vspace{1.5cm}
\begin{figure}[h!]
\centering\includegraphics[scale=0.45]{GANTT2.png}
\caption{\label{fig:GANTT} Diagramme de Gantt du système BMONS}
\end{figure}


% WBS et diagramme de Gantt


%----------------------------------------------------------------------------------------
%	PART VI 
%----------------------------------------------------------------------------------------
\part{Journal du projet}
\chapter{Choix et justifications}

\section{Choix des capteurs}

Après avoir réalisé l'état de l'art pour notre système de surveillance d'une ruche, nous nous sommes ensuite intéressés aux capteurs que nous allons employer.\\

Capteur de température\\

Nous avons choisis un capteur de température identique pour recueillir la température interne et externe de la ruche.
Il s'agit d'une thermistance NTC boitier Goutte Radial 1000 ohms sortie fil de cuivre émaillé 1 pc(s). Ce dernier possède une gamme de mesure comprise entre -40 C et 100 C ce qui correspond bien aux exigences discutées avec le client (Voir tableau des exigences). Le capteur placé à l'extérieur nous renseignera sur les conditions météorologique et expliquer une éventuelle diminution de la production de miel par exemple. Celui placé à l'intérieur nous donnera une idée de l'isolation de la ruche en hiver et de sa ventilation en été.\\  

Capteur d'humidité\\

On a choisit d'inclure ce capteur dans notre système compte tenu des résultats de l'état de l'art. Néanmoins, après discussion avec le client, cette option n'est pas primordiale pour un apiculteur mais elle sera tout de même rajoutée au tableau de bord du serveur.\\

Capteur de pression\\

Les capteurs de pression vont nous permettre de récupérer le poids de ruche et surtout celui des hausses pour avertir l'apiculteur de la quantité de miel produite. Pour se faire, le projet Bzzz développé par le Fablab de Lannion a prévu d'utiliser deux sachets de "Pompote" remplis d'eau sucrée pour éviter les variations de pression atmosphérique, le gel et l'évaporation. Cependant, après avoir pris conscience de l'importance de la localisation de la grappe, nous avons pensé installer deux capteurs de pression par cadre (au niveau de chaque extrémité) soit 20 au total mais cette solution s'est avérée être difficile à mettre en place à cause de la surface sur laquelle repose les cadres (simple lamelle en métal). Après discussion avec le client, nous avons finalement opté pour la confection d'un cadre en bois aux même dimensions de la ruche dans lequel se trouverons les capteurs de pression. L'utilisateur décidera de l'endroit où le placer en fonction des données qu'il veut récupérer. 
On pourra aussi utiliser ce capteur pour détecter l'ouverture du couvercle à cause du vent.\\

Tilt sensor\\

Ce capteur permet de savoir si la ruche a reçu un choc ou si elle a été déplacé. Il renvoi une information binaire qui pourra être couplée avec les données de la carte GPS et ainsi avertir l'apiculteur en cas de détérioration, renversement ou vol de la ruche. Ce dernier évènement étant de plus en plus fréquent.
Nous avons choisit le TikerKit Tilt Sensor.\\

Microphone\\

L'utilisation d'un microphone s'est imposée comme une nécessité dans la surveillance d'une ruche car il permet de recueillir des données pouvant avertir l'apiculteur sur plusieurs types d'évènements. En effet, grâce à ce dernier, on pourra détecter la présence d'abeilles dans la ruche notamment en hiver et éventuellement recueillir la source du bourdonnement et ainsi localiser la grappe approximativement. On pourra aussi détecter les prémisses d'un essaimage en percevant le champs d'une raine caractéristique de ce type d'évènement. Cette information pourra ensuite être couplée avec les données des capteurs de pression et confirmer l'essaimage en cours si le poids chute brutalement. 
Nous avons choisit le Capsule micro pour CI 2 V/DC sensibilité 44 dB (à 3 dB près) sur la plage 100 - 10000 Hz. 

\chapter{Résultats et analyses}

analyse des tests et des performances
analyse des échecs, des décalages et des retards
Que reste-t-il à faire ? Comment ?


\chapter{Conclusion}



\appendix
\part{Annexes}
\chapter{Première annexe: Fonctions métier du système}

\begin{figure}[h!]
\centering\includegraphics[trim = 1cm 2cm 1cm 1cm,scale=0.8]{Exigences1.pdf}
\caption{\label{fig:exi1} Fonctions métiers du système (1/4)}
\end{figure}

\begin{figure}[h!]
\centering\includegraphics[trim = 1cm 2cm 1cm 1cm,scale=0.8]{Exigences2.pdf}
\caption{\label{fig:exi2} Fonctions métiers du système (2/4)}
\end{figure}

\begin{figure}[h!]
\centering\includegraphics[trim = 1cm 2cm 1cm 1cm,scale=0.8]{Exigences3.pdf}
\caption{\label{fig:exi3} Fonctions métiers du système (3/4)}
\end{figure}

\begin{figure}[h!]
\centering\includegraphics[trim = 1cm 2cm 1cm 1cm,scale=0.8]{Exigences4.pdf}
\caption{\label{fig:exi4} Fonctions métiers du système (4/4)}
\end{figure}

\chapter{Deuxième annexe: Diagramme Flux de données}

\begin{figure}[h!]
\centering\includegraphics[scale=0.35]{diagFluxDonnees.jpg}
\caption{\label{fig:donnees} Diagramme d'activité du flux de données}
\end{figure}

\chapter{Troisième annexe: Spécification fonctionnelle: Axe Data}

\begin{figure}[h!]
\centering\includegraphics[scale=0.8]{axeData1.png}
\caption{\label{fig:axeData1} Axe Data (1/3)}
\end{figure}

\begin{figure}[h!]
\centering\includegraphics[scale=0.8]{axeData2.png}
\caption{\label{fig:axeData2} Axe Data (2/3)}
\end{figure}

\begin{figure}[h!]
\centering\includegraphics[scale=0.8]{axeData3.png}
\caption{\label{fig:axeData3} Axe Data (3/3)}
\end{figure}

\backmatter % Partie finale du document, non numérotée

%----------------------------------------------------------------------------------------
%	BIBLIOGRAPHIE
%----------------------------------------------------------------------------------------
\addcontentsline{toc}{part}{Bibliographie}
%\bibliographystyle{apalike-fr}
\bibliographystyle{plain-fr}
\bibliography{bibliographie}
\nocite{*}

%----------------------------------------------------------------------------------------
%	INDEX
%----------------------------------------------------------------------------------------
\cleardoublepage
\phantomsection
\setlength{\columnsep}{0.75cm}
\addcontentsline{toc}{part}{Index}
\label{sec:index}
\printindex

%----------------------------------------------------------------------------------------
%	GLOSSAIRE
%----------------------------------------------------------------------------------------
\cleardoublepage
\phantomsection
\setlength{\columnsep}{0.75cm}
\addcontentsline{toc}{part}{Glossaire}
\printglossaries

%----------------------------------------------------------------------------------------

\end{document}
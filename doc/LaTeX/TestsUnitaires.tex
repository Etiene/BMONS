\chapter{Tests}

\underline{Test de résistance à la cire d'un capteur de température}\newline 

Matériel:  
3 capteurs de température  
Cire d'abeilles  
1 support\newline   

Préparation:  
Fixer les capteurs de température sur le support espacés de manière régulière sur le capteur de gauche appliquer 0.5 cm de cire sur le capteur de droite appliquer 1 cm de cire le capteur central sera le témoin \newline  

Test 1: \newline  
Mettre le support au frigo et surveiller l'évolution de la température\newline   
Relever le temps nécessaire pour arriver à la même température que le témoin pour les 2 capteurs tests  

Test 2:  \newline 
Directement après le test 1 enlever le support du frigo et surveiller l'évolution de la température\newline   
Relever le temps nécessaire pour arriver à la même température que le témoin pour les 2 capteurs tests

Test 3:  \newline 
Mettre le support dans un "four" à 30 degrés et surveiller l'évolution de la température\newline   
Relever le temps nécessaire pour arriver à la même température que le témoin pour les 2 capteurs tests  

Test 4:  \newline 
Directement après le test 3 enlever le support du "four" et surveiller l'évolution de la température\newline   
Relever le temps nécessaire pour arriver à la même température que le témoin pour les 2 capteurs tests \newline 

\underline{Test Capteur d'humidité}\newline 

Matériel: 
1 capteur d'humidité des sachets déshydratants\newline  
2 boites/sachets hermétiques cotons et eau \newline 

Préparation: 
Commencer et finir le cycle avec un capteur propre \newline 

Etape 0:  

Relever le niveaux d'humidité d'origine\newline  

Etape 1: 
Mettre le support dans réceptacle à 100\% d'humidité\newline  
Surveiller l'évolution\newline  
Relever les courbes\newline  

Etape 2: 

Mettre le support dans un réceptacle à 0\% d'humidité\newline  

Etape 3:  
Rsépéter les  étapes 1 et 2 cinq fois avec différentes épaisseurs
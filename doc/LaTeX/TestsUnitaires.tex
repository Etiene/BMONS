\chapter{Tests}

\section{Test de résistance à la cire d'un capteur de température}

\noindent \underline{Matériel}:  
3 capteurs de température  
Cire d'abeilles  
1 support\newline  

\noindent \underline{Préparation:} \newline 
Fixer les capteurs de température sur le support espacés de manière régulière sur le capteur de gauche appliquer 0.5 cm de cire sur le capteur de droite appliquer 1 cm de cire le capteur central sera le témoin \newline

\noindent \underline{Test 1:} \newline 
Mettre le support au frigo et surveiller l'évolution de la température\newline   
Relever le temps nécessaire pour arriver à la même température que le témoin pour les 2 capteurs tests  

\noindent \underline{Test 2:}  \newline 
Directement après le test 1 enlever le support du frigo et surveiller l'évolution de la température\newline   
Relever le temps nécessaire pour arriver à la même température que le témoin pour les 2 capteurs tests

\noindent \underline{Test 3:}  \newline 
Mettre le support dans un "four" à 30 degrés et surveiller l'évolution de la température\newline   
Relever le temps nécessaire pour arriver à la même température que le témoin pour les 2 capteurs tests  
  
\noindent \underline{Test 4:}  \newline 
Directement après le test 3 enlever le support du "four" et surveiller l'évolution de la température\newline   
Relever le temps nécessaire pour arriver à la même température que le témoin pour les 2 capteurs tests


\newpage

\section{Test Capteur d'humidité} 

\noindent \underline{Matériel}: 
1 capteur d'humidité des sachets déshydratants\newline  
2 boites/sachets hermétiques cotons et eau \newline 

\noindent \underline{Préparation:} \newline  
Commencer et finir le cycle avec un capteur propre\newline  

\noindent \underline{Etape 0:}  

\noindent Relever le niveaux d'humidité d'origine  

\noindent \underline{Etape 1:} \newline  
Mettre le support dans réceptacle à 100\% d'humidité\newline  
Surveiller l'évolution\newline  
Relever les courbes

\noindent \underline{Etape 2:} 

\noindent Mettre le support dans un réceptacle à 0\% d'humidité 

\noindent \underline{Etape 3:}  \newline
Répéter les  étapes 1 et 2 cinq fois avec différentes épaisseurs

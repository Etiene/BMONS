\chapter{Méthodes de travail}

% Méthodes de travail
% Organisation temporelle, spatiale, humaine 
% interactions des membres de l’équipe projet
% interactions avec les encadrants
% interactions avec les tiers

Tout au long du projet notre méthode de travail a changée. Au fur et à mesure que les séances s'enchainaient, et en prenant en compte les conseils qui nous ont été donnés (notamment à propos du carnet de bord et des objectifs à court terme) notre méthode de travail a tendu vers la suivante : \\ \\
De manière générale, toute l'équipe du projet BMONS travaille dans la même salle pour faciliter la communication entre les membres du groupe. Une séance de travail commence par l'ouverture personnelle des mails de chacun, puis le groupe se réunit pour définir les objectifs de la matinée, ensuite chacun choisit la partie il va avancer. Le travail se fait en général seul ou en binôme et des points d'avancement sont faits à l'oral tout au long de la séance. Parfois des tâches comme la prise en main d'un logiciel ou la compréhension d'un diagramme sont faites en dehors des séances, mais la majeure partie du travail s'effectue lors du temps alloué au projet. \\ \\
Les réunions avec les encadrants et les intervenants extérieurs se déroulent dans des salles de l'ENSTA Bretagne équipées d'un vidéo projecteur et en présence de la totalité de l'équipe BMONS. Ces séances sont organisées à l'avance, les points sur lesquels des précisions sont nécessaires sont mis en avant avant la séance et des questions précises sont préparées. Cela permet de guider la réunion et de ne pas perdre de temps sur des points déjà vus. Chacun a son rôle lors de ces réunions, la prise de notes, le dialogue avec l'intervenant et la rédaction du compte rendu sont ainsi facilités.


 

\chapter{Outils pour les échanges}

% Quels sont les outils qui nous permettent de travailler ensemble ?

Les outils qui nous ont permis de travailler ensemble et de partager nos fichiers ont également changés au cours du temps. 
Avant les premiers ateliers techniques, particulièrement celui sur github, nous partagions nos résultats sur le oneDrive d'office 365. \\

\chapter{Répartition des tâches dans le temps}


% WBS et diagramme de Gantt

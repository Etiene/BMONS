\chapter{Conclusion}\\

Le projet BMONS a pour ambition de réaliser un système embarqué sur une ruche pour récolter les paramètres physiques de la ruche dans l’optique de détecter les anomalies de comportement des abeilles. Notre équipe s’est donc penché sur cette problématique avec des notions d’apiculture et de conduite de projet plutôt minces, voici un premier bilan.

Après cette première partie du projet nous pouvons enfin dire que nous comprenons les besoins et les attentes des apiculteurs, par exemple des fonctions auxquelles nous avions pensé, comme la détection du taux d'humidité dans la ruche ne semblent finalement pas primordiale au projet. En revanche des options comme le comptage des abeilles seraient appréciées par les apiculteurs, mais des compromis doivent êtres faits car certaines de ces options entrainent un surcoût trop conséquent pour que le projet satisfasse les exigences initiales.\newline 

Nous devons une grande partie de ces connaissances en apiculture aux réponses précises de monsieur Franck Singhoff, qui nous a également prêté une ruche pour que nous puissions travailler sur l'implémentation des capteurs sur notre prototype. \newline 

Les difficultés que nous avons rencontrées, par exemple lors de la compréhension et de la création des diagrammes nécessaires à l’établissement de la spécification fonctionnelle sur les trois axes, ont été surmontées grâce à l’aide de nos encadrants de projet, mais aussi grâce au dialogue entre les membres du groupe. Les choix des divers composants qui constitueront le prototype ont aussi été une source de problèmes, car même si nous avions une idée de quel type de composant nous avions besoin, faire le tri entre tous ceux qui existent nécessite de faire des choix entre par exemple le prix et la précision d’un capteur, et cela influe sur les exigences du projet. \newline

Ces quelques mois de travail en commun nous ont permis d'avoir un aperçu d'ensemble de la conduite d'un projet. Nous nous sommes rendu compte de la quantité de travail que représente la partie ingénierie système sur un projet comme celui-ci. Et bien que nous ne puissions pas prévoir tous les aléas de la seconde partie du projet, ce travail d'anticipation nous permet d'être plus sereins face au travail qu'il reste à fournir.\newline​



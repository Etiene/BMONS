\chapter{Formulation initiale du projet}



\section{Contexte}

BeeHive Monitoring System (BMONS) est un projet qui a pour but d'aider les apiculteurs. Il s'agit de leur proposer un système de surveillance et de détection peu onéreux afin de prodiguer les meilleurs soins au meilleur moment aux ruches qui en ont besoin et d'éviter les vols.

En effet, les abeilles sont vitales à l'équilibre écologique. Einstein avait même dit: " Si l’abeille disparaît, l’humanité en a pour quatre ans à vivre ". Sans elles 84 \% des espèces végétales cultivées pour l'alimentation disparaitraient. Or les abeilles sauvages sont aujourd'hui rares et l'espèce ne survivrai pas sans l'aide des apiculteurs. Ainsi le travail de ces derniers est crucial non seulement pour assurer la production de miel mais aussi pour la sauvegarde de l'environnement. Cependant, ces dernières années, les apiculteurs ont été confrontés à de nombreux problèmes et nous sommes aujourd'hui face à une diminution du nombre d'abeilles telles que la production annuelle européenne de miel est quatre fois moindre que celle qu'il y a vingt ans. 

Pour aider à la résolution de ce problème, nous voulons donc créer un système capable d'aider l'apiculteur dans son travail et de ce fait combattre la disparition des abeilles. 

\section{Expression initiale du besoin}

Après avoir discuté avec plusieurs apiculteurs, nous avons pu identifier leurs besoins et déterminer de quelle manière nous pouvons les aider. Ainsi l'objectif de ce système est tout d'abord de donner accès à l'apiculteur à des informations clés sur la ruche sans que celui-ci n'ai à se déplacer, ni à ouvrir les ruches. En effet l'ouverture de la ruche perturbe les abeilles et elle n'est pas possible en hiver à cause des températures trop basses. De plus les ruches sont souvent disposées dans des ruchers éloignés les uns des autres, ce qui complique le travail de l'apiculteur. Les informations nécessaires seraient : la température dans et en dehors de la ruche, le poids, l'humidité et les sons de la ruche. Mais le système devra aussi alerter l'apiculteur quand la sécurité de la ruche est compromise, pour permettre une action rapide destinée à sauver la colonie.

Le système BMONS est donc composé de deux parties distinctes. La première consiste en un élément embarqué dans la ruche qui consomme un minimum d'énergie et qui mesure les paramètres clés. Les données de cet élément embarqué sont transmises via un transmetteur sans fils à un serveur qui constitue la deuxième partie du système. Il donne accès à l'apiculteur aux différentes mesures effectuées dans et autour des ruches. Il envoie également des alertes de sécurités à l'apiculteur si besoin. 

